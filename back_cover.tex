\newgeometry{margin=1.5in}
\begin{titlepage}
\begin{tikzpicture}[remember picture, overlay]
  \fill[top color = upper, bottom color = lower] (current page.north west) rectangle (current page.south east);
\end{tikzpicture}
\ltx{} is a software system for typesetting documents. Because it is especially good for technical 
documents and is available for almost any computer system, \ltx{} has become a lingua franca of the 
scientific world. Researchers, educators, and students in universities, as well as scientists in 
industry, use \ltx{} to produce professionally formatted papers, proposals, and books. They also 
use \ltx{} input to communicate information electronically to their colleagues around the world

With the release of \ltxee{}, the new standard version, \ltx{} has become even more powerful. 
Among its new features are an improved method for handling different styles of type, and commands 
for including graphics and producing colors. \ltxee{} makes available to 
all \ltx{} users valuable enhancements to the software that have been developed over the years 
by users in many different places to satisfy a variety of needs. 

This book, written by the original architect and implementer of \ltx{}, is both the user's guide
and the reference manual for the software. It has been updated to reflect the changes in the new
release. The book begins with instructions for formatting simpler text, and progreely decribes
commands and techniques for handling larger and more complicated documents. A separate chapter
explains how to deal with errors. An added appendix describes what is new and different in \ltxee{}.
Other additions to the second edition include:

\begin{itemize}
  \item Descriptions of new commands for,inserting pictures prepued with other programs
    and for producing colored output;
  \item New sections on how to make books and slides;
  \item Instructions for making an index with the \textit{MakeIndex} program, and an updated 
    guide to preparing for a bibliography with the \hologo{BibTeX} program;
  \item A section on how to send your \ltx{} documents electronicaly.
\end{itemize}

Users new to \ltx{} will find here a book that has earned worldwide praise as a model for clear,
concise, and practical documentation. Experienced users will want to update their \ltx{} library. 
Although most standard \ltx{} input files will work with \ltxee{}, to take advantage of
the new features, a few \ltxee{} conventions must first be learned. For users who 
want an advanced guide to \ltxee{} and to more than 150 packages that can now be used 
at any site to provide additional features, a useful companion to this book is \textit{The \ltx{} Companion}, 
by Goossens, Mittelbach, and Samarin (also published by Addison-Wesley).

\medskip
\noindent\parbox{.6\linewidth}{
\qquad Leslie Lamport is a computer scientist well known for
his contributions to concurrent computing, as well as for
creating the \ltx{} typesetting system in 1985. He now
works at the Systems Research Center of Digital Equip
ment Corporation. He received a Ph.D.in mathematics
from Brandeis University.

\bigskip
\includegraphics[width=1.6em]{./figure/addison-wesley-logo.png} %
ADDISON-WEILY PUBLISHING COMPANY
}\parbox{.4\linewidth}{\hfill\scalebox{.35}{\ITFbarcode[frame=false, S1.2]{97802015298310}}}
\end{titlepage}
\restoregeometry