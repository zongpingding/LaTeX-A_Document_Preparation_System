\section{How to Read This Book}
While \texttt{sample2e.tex} illustrates many of \ltx{}'s features, it is still only about
two hundred lines long, and there is a lot that it doesn't explain. Eventually,
you will want to typeset a document that requires more sophisticated formatting
than you can obtain by imitating the two sample input files. You will then have
to look in this book for the necessary information. You can read the section
containing the information you need without having to read everything that
precedes it. However, all the later chapters assume you have read Chapters 1
and 2. For example, suppose you want to set one paragraph of a document in
small type. Looking up ``type size7'' in the index or browsing through the table
of contents will lead you to Section 6.7.1, which talks about ``declarations'' and
their ``scope''-simple concepts that are explained in Chapter 2. It will take just
a minute or two to learn what to do if you've already read Chapter 2; it could
be quite frustrating if you haven't. So, it's best to read the first two chapters
now, before you need them.


\ltx{}'s input is a file containing the document's text together with com-
mands that describe the document's structure; its output is a file of typesetting
instructions. Another program must be run to convert these instructions into
printed output. With a high-resolution printer, \ltx{} can generate book-quality
typesetting.

This book tells you how to prepare a \ltx{}input file. The current chapter
discusses the philosophy underlying \ltx{}; here is a brief sketch of what's in the
remaining chapters and appendices:

\def\listSketch#1#2{%
\bigskip\noindent\textbf{#1}~#2\par
}

\listSketch{Chapter 2}{explains what you should know to handle most simple documents
and to read the rest of the book. Section 2.5 contains a summary of
everything in the chapter; it serves as a short reference manual.}

\listSketch{Chapter 3}{describes logical structures for handling a variety of formatting
problems. Section 3.4 explains how to define your own commands, which
can save typing when you write the document and retyping when you
change it. It's a good idea to read the introduction-up
to the beginning of Section 3.1-before reading any other part of the chapter.}

\listSketch{Chapter 4}{contains features especially useful for large documents, including
  automatic cross-referencing and commands for splitting a large file into
  smaller pieces. Section 4.7 discusses sending your document electronically.}

\listSketch{Chapter 5}{is about making books, slides, and letters (the kind you send by post).}

\listSketch{Chapter 6}{describes the visual formatting of the text. It has information about
  changing the style of your document, explains how to correct bad line and
  page breaks, and tells how to do your own formatting of structures not
  explicitly handled by \ltx{}.}

\listSketch{Chapter 7}{discusses pictures--drawing them yourself and inserting ones prepared with other 
  programs--and color.}

\listSketch{Chapter 8}{explains how to deal with errors. This is where you should look when \ltx{} prints an 
error message that you don't understand.}

\listSketch{Appendix A}{describes how to use the \textit{MakeIndex} program to make an index.}

\listSketch{Appendix B}{describes how to make a bibliographic database for use with
\hologo{BibTeX}, a separate program that provides an automatic bibliography feature for \ltx{}}

\listSketch{Appendix C}{is a reference manual that compactly describes all \ltx{}'s features, 
including many advanced ones not described in the main text. If a command introduced in the 
earlier chapters seems to lack some necessary capabilities, check its description here to see if it has them. 
This appendix is a convenient place to refresh your memory of how something works.}

\listSketch{Appendix D}{describes the differences between the current version of \ltx{}
and the original version, \ltx{} 2.09.}

\listSketch{Appendix E}{is for the reader who knows \TeX{}, the program on which \ltx{} built, and wants to use 
\TeX{} commands that are not described in this book.}

When you face a formatting problem, the best place to look for a solution is in
the table of contents. Browsing through it will give you a good idea of what
B W has to offer. If the table of contents doesn't work, look in the index; I
have tried to make it friendly and informative.

Each section of Chapters 3-7 is reasonably self-contained, assurning only
that you have read Chapter 2. Where additional knowledge is required, explicit
cross-references are given. Appendix C is also self-contained, but a command's
description may be hard to understand without first reading the corresponding
description in the earlier chapters.

The descriptions of most commands include examples of their use. In this book, examples are 
formatted in two columns, as follows:

\bigskip
\noindent\parbox{.45\linewidth}{The left column shows the printed output; the right column contains the input that produced it.}
\hspace*{2em}
\parbox{.65\linewidth}{\ttfamily The left column shows the printed output; the right column contains the input that produced it.}
\bigskip

Note the special typewriter type style in the right column. It indicates what
you type-either text that you put in the input file or something like a file name
that you type as part of a command to the computer.

Since the sample output is printed in a narrower column, and with smaller
type, than \ltx{} normally uses, it won't look exactly like the output you'd get
from that input. The convention of the output appearing to the left of the
corresponding input is generally also used when commands and their output are
listed in tables.