\ltx{} is a system for typesetting documents. Its first widely available version,
mysterjiously numbered 2.09, appeared in 1985. \ltx{} is now extremely popular
in the hientific and academic communities, and it is used extensively in industry.
It has become a lingua frmnca of the scientific world; scientists send their papers
electropically to colleagues around the world in the form of \ltx{} input.


Over the years, various nonstandard enhancements were made to \ltx{} 2.09
to oveqcome some of its limitations. \ltx{} input that made use of these en-
hance~ents would not work properly at all sites. A new version of \ltx{} was
needed to keep a Tower of Babel from rising. The current version of \ltx{},
with the so~newhat less mysterious number $2\varepsilon$, was released in 1994. \ltxee{}
contains an improved method for handling different styles of type, commands
for including graphics and producing colors, and many other new features.


Almost all standard \ltx{} 2.09 input files will work with \ltxee{}. However,
to take advantage of the new features, users must learn a few new \ltxee{}
conventions. \ltx{} 2.09 users should read Appendix D to find out what has
changed. The rest of this book is about \ltx{}, which, until a newer version
appears, means \ltxee{}.

\ltx{} is available for just about any computer made today. The versions
that rup on these different systems are essentially the same; an input file created
according to the directions in this book should produce the same output with
any of them. However, how you actually run \ltx{} depends upon the computer
system. Moreover, some new features may not be available on all systems when
\ltxee{} is first released. For each computer system, there is a short companion
to this book, titled something like {\itshape Local Guide to \ltx{} for the McKludge PC},
containing information specific to that system. I will call it simply the {\itshape Local Guide}. 
It is distributed with the \ltx{} software.

Thete is another companion to this book, {\itshape The \ltx{} Companion} by Goossens,
Mittelbach, and Samarin\cite{goossens1994companion}. This companion is an in-depth guide to \ltx{} and
to its \textit{packages}-standard enhancements that can be used at any site to provide
additional features. The \ltx{} Companion is the place to look if you can't find what you 
need in this book. It describes more than a hundred packages.