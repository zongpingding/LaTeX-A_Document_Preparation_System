\section{How to Avoid Reading This Book}
Many people would rather learn about a program at their computer than by reading a book. There is a small 
sample \ltx{} input file named \texttt{small2e.tex} that shdws how to prepare your own input files for 
typesetting simple documents. Before reading any further, you might want to examine \texttt{small2e.tex} with a text
editor and modify it to make an input file for a document of your own, then run \ltx{} on this file and see 
what it produces. The Local Guide will tell you how to find \texttt{small2e.tex} and run \ltx{} your computer; 
it may also contain information about text editors. Be careful not to destroy the original version of
\texttt{small2e.tex}; you'll probably want to look at it again.


The file \texttt{small2e.tex} is only forty lines long, and it shows how to produce
only very simple documents. There is a longer file named \texttt{small2e.tex} that
contains more information. If \texttt{small2e.tex} doesn't tell you how to do some-
thing, you can try looking at \texttt{small2e.tex}.

If you prefer to learn more about a program before you use it, read on.
Almost everything in the sample input files is explained in the first two chapters
of this book.