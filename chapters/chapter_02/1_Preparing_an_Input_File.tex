\section{Preparing an Input File}
The input to \ltx{} is a text file. I assume that you know how to use a text
editor to create such a file, so I will tell you only what should go into your input
file, not how to get it there. A good text editor can be customized by the user
to make it easier to prepare \ltx{} input files. Consult the Local Guide to find
out how to customize the text editors on your computer.

On most computers, file names have two parts separated by a period, like
\texttt{sample2e.tex}. I will call the first part its first name and the second part 
its \textit{extension}, so \texttt{sample2e} is the first name of \texttt{sample2e.tex}, and 
\texttt{tex} is its extension. Your input file's first name can be any name allowed by your 
computer system, but its extension should be \texttt{tex}.

Let's examine the characters that can appear in input file. First, there
are the upper- and lowercase letters and the ten digits \texttt{0 ... 9}. Don't confuse
the uppercase letter \texttt{O}(oh) with the digit \texttt{0}(zero), or the letter \texttt{l}(the lowercase
el) with the digit \texttt{1}(one). Next, there are the following sixteen punctuation
characters:

\begin{center}
  \ttfamily
  . : ; , ? ! ` ' ( ) [ ] - / * @
\end{center}

Note that there are two different quote symbols: ` and ' . You may think of ' as
an ordinary ``single quote'' and ` as a funny symbol, perhaps displayed like \texttt{\textasciigrave} on
your screen. The \textit{Local Guide} should tell where to find ` and ' on your keyboard,
if they're not obvious. The characters \texttt{(} and \texttt{)} are ordinary parentheses, while
\texttt{[} and \texttt{]} are called square brackets, or simply brackets.

The ten special characters
\begin{center}
  \ttfamily
  \# \$ \% \& \_ \{ \} \~{} \^{} \textbackslash
\end{center}

are used only in \ltx{} commands. Check the \textit{Local Guide} for help in finding
them on your keyboard. The character \verb|\| is called \textit{backslash}, and should not
be confused with the more familiar /, as in 1/2. Most B\ltx{} commands begin
with a \verb|\| character, so you will soon become very familiar with it. The \verb|{| and \verb|}|
characters are called \textit{curly braces}\index{curly braces} or simply \textit{braces}